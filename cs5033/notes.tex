\documentclass{sty/SizheArticle}

\title{Revision notes}
\author{140011146}
\addbibresource{references.bib}
\usepackage{sty/sizhetitle}
\usepackage{amsthm}
\usepackage{multicol}

\newtheorem{definition}{Definition}

\begin{document}


\maketitlepage{CS5033 Software Architecture}{Dharini Balasubramaniam}
\tableofcontents

\section{Introduction}
\subsection{Analogy to architecture of buildings}
There is a strong analogy between the architecture of software systems and the
architecture of buildings. Requirements are gathered, a design is created to
satisfy the requirements, the design is refined to create blueprints and
the building is constructed following the blueprints, to be used and occupied.
In software systems, requirements are specified, a high-level design is
created, code is written to implement the design and the system is deployed
and used. This analogy offers several insights.
\begin{enumerate}
\item The concept of a building having an architecture that is a separate form
which is liked to the physical structure itself. The architecture during the
design can be compared to the architecture of the physical structure after
construction. In a software system, that architecture should also exist in
an independent form, linked to the code that implements the architecture.
\item The properties of structures are induced by the design of their
architectures, for example a castle with thick walls is designed to have
defensive properties against attacks. In a software system, properties such
as resilience against security attacks are determined by the design of the
architecture.
\item There is a distinctive role and character of the architect, separate
from that of the construction workers. In the analogy, this means that skill
in programming is not sufficient for the creation and conception of complex
software systems.
\item The process is not as important as the architecture. Simply following
a standard process will not guarantee that a successful building will be
created which meets the original requirements. Architects and engineers must
keep its design and qualities in mind and a process is only present to serve 
those ends, not be an end in itself.
\item Architecture develops from existing knowledge and previous work. New
buildings are not designed from first principles and re-discovery of materials.
\end{enumerate}

We also see the goals of building architecture and mostly similar to those of
software architecture.
\begin{itemize}
\item Good quality
\item On time
\item On budget
\item Satisfies customer requirements
\end{itemize}

\subsection{Definitions}
There are many different definitions of software architecture:

\begin{definition}
\textbf{Software architecture} is the fundamental organisation of a system 
embodied in its components, their relationships to each other and 
to the environment and the principles guiding its design and evolution. 
[IEEE 1471 - 2000]
\end{definition}

\begin{definition}
\textbf{Software architecture} is the structure or structures of the system,
which comprise software elements, the externally visibly properties of
these elements, and the relationships among them. [Bass, Clements and Kazman]
\end{definition}

\begin{definition}
\textbf{Software architecture} comprises of descriptions of elements from which
systems are built, interactions among these elements, pattern that guide their
composition, and constraints on these patterns. [Shaw and Garlan]
\end{definition}

The main motivation for software architecture is to have a high-level
specification of the system which captures important properties and can be
used as a basis for development and management activities.  A software 
architecture also allows the design to be verified or analysed, either
manually or automatically. Software architecture can serve as a basis for
a variety of areas:
\begin{itemize}
\item Recognition of common problem classes
\item Separation of concerns and complexity management
\item Satisfying requirements (functional and non-functional)
\item Development focus
\item Stakeholder communication
\item System design and implementation
\item Maintenance and evolution
\item Reuse
\item Cost estimation
\item Project and process management
\end{itemize}

\subsection{Fundamental concepts}
There are two aspects to software architecture that key concepts fall under:
the system and the stakeholders. 
\begin{enumerate}
\item System
\begin{itemize}
\item Components
\item Connectors (interactions)
\item Configurations
\item Properties
\item Styles/patterns
\item Rationale
\item Interfaces
\end{itemize}
\item Stakeholders
\begin{itemize}
\item Views and viewpoints
\item Requirements (functional and non-functional)
\item Rationale (again)
\item Priorities
\item Risks
\item Tradeoffs
\end{itemize}
\end{enumerate}

\subsubsection{Component}
\begin{definition}
A software component is an architectural entity that (1) encapsulates a subset
of the system's functionality and/or data, (2) restricts access to that subset
via an explicity defined interface, and (3) has explicitly defined dependencies 
on its required execution context.
\end{definition}

Components are the building block of architectures. These are typically units
of computation or a data store (processing and data elements). They encapsulate
the processing and data in a system's architecture.
Examples are things like a server, database, client, etc.
The extent of the context captured by a component can include:
\begin{itemize}
\item The components required \textbf{interface} to services provided by other
components in a system that this components depends on to operate.
\item The availability of specific resources, such as data files that the
components relies on
\item The required software environment, such as programming language, middleware
and operating system
\item The hardware configurations needed to execute the component
\end{itemize}

An important aspect to software components is making sure they are reusable, as
a software architecture comprises of many components with possible overlap. 
This has implications for how the problem solve, how they are distributed
and how they are allocated. A suitable decomposition strategy is critical
to components of a software architecture.
\subsubsection{Connector}
\begin{definition}
A software connector is an architectural elements tasked with effecting and
regulating interactions among components.
\end{definition}

Connectors are the elements that deal with the interaction among a system's
components. They model the rules that govern the interactions, with functionality
include
\begin{itemize}
\item Communication
\item Coordination
\item Conversion
\item Facilitation
\end{itemize}

Examples of connectors include procedure calls, shared memory, pipes,
synchronous message passing, etc. They can be either simple or semantically
very rich. Due to a shift towards microservice architectures, connectors
are increasingly important as the facilitate the interaction between
the microservices.

Note that while components provide application-specific services, connectors
are typically application-independent. The characteristics of a procedure call,
distributor, adaptor, etc. are independent of the components they service.

\subsubsection{Configuration}
\begin{definition}
An architectural configuration is a set of specific associations between
the components and connectors of a software architecture.
\end{definition}

Configurations are used to define the topology of the system. A software
system can be viewed as a connected graph of components and edges and
the configuration defines this graph. The configuration can capture aspects
such as concurrency and distribution. Some composite elements may already
have a set configuration. The aim of having configurations is to minimise
dependencies.

\subsubsection{Architectural styles and patterns}
\begin{definition}
An architectural style is a named collection of architectural design decisions
that (1) are applicable in a given development context, (2) constrain
architectural design decisions that are specific to a particular system within
that context, and (3) elicit beneficial qualities in each resulting system.
\end{definition}

Architectural styles are reusable architectural templates. The idea is that
many existing and different systems may all have similar design choices.
These solutions have been evolved over time and carried on to be more
elegant, efficient, scalable, etc. compared to creating a new architecture
from scratch. Multiple of these patterns can be combined in a single system.
Each pattern implies the existence of some components and interactions, for
example a client-server implies a single server and multiple client components.

\subsubsection{Rationale}
The rationale is an important aspect to the architecture, which is about
why the system should be designed. This is linked heavily to customer
requirements for an architecture, or the various intent and assumptions
during the design process.

\subsubsection{Properties}
Properties of a software architecture are the non-functional requirements
and quality attributes that define the performance or other requirements of
the system. Ideally, these properties are quantified and verifiable, for example
throughput in messages per second. 

\section{Software architecture styles}
There are a number of different domains and problems which have solutions
with similar structure and properties. Architecture styles capture the
experience and learning from previous systems into a reusable design for
new systems. There are a few advantages to using architecture styles
\begin{itemize}
\item Reduced development time and cost - Less time is spent designing and
analysing the system because there are inherent properties to some
architectural styles
\item Improved system quality - Again, the properties of the chosen
architectural style will guarantee some system properties, which helps
evaluate system quality on a high-level
\end{itemize}

The notion of architectural style is useful from both descriptive and prescriptive
points of view. \textbf{Descriptively}, architectural style defines a
particular codification of design elements and formal arrangements.
Prescriptively, style limits the kinds of design elements and their formal
arrangements. So the style can constrain both the design elements and the
relationships between the design elements.

There is also a differentiation that should be made between \textbf{design}
patterns and \textbf{architectural patterns}. Design patterns provide
a template for subsystems or components of a system, but not a template
for the entire system itself. This is a difference in the scale of each
type of pattern. Design patterns do not influence the fundamental structure
of a software system and only affect a single subsystem. In other words,
it can be said that design patterns help implement architectural patterns.

Example styles:
\begin{itemize}
\item \textbf{Layered}
	\begin{itemize}
	\item Virtual machines
	\item Client-server
	\end{itemize}
\item \textbf{Dataflow styles}
	\begin{itemize}
	\item Batch-sequential
	\item Pipe-and-filter
	\end{itemize}
\item \textbf{Shared memory}
	\begin{itemize}
	\item Blackboard
	\item Rule-based
	\end{itemize}
\item \textbf{Implicit invocation}
	\begin{itemize}
	\item Publish-subscribe
	\item Event-based
	\end{itemize}
\end{itemize}

Note that a style or pattern does not entirely define an architecture, but
rather a family of architectures which obey a common set of constraints.
Most architectures are heterogeneous, which uses a combination of patterns,
models and technologies.

\subsection{Style selection}

\renewcommand{\arraystretch}{1.75}
\begin{table}[H]
\begin{tabular}{p{0.2\textwidth} p{0.25\textwidth} p{0.25\textwidth} p{0.25\textwidth}}
\textit{Style name} & \textit{Summary} & \textit{Use it when...} & \textit{Avoid it when...} \\ \hline
\multicolumn{4}{l}{\textbf{Language-influenced styles}} \\
Main program and subroutines & Main program controls program execution,
calling multiple subroutines & ...application is small and simple & 
...complex data structures are needed

...future modifications are likely \\
Object-oriented & Objects encapsulate state and accessing functions &
...close mapping between external entities and internal objects is sensible

...many complex and interrelated data structures & ...strong independence between
components necessary, very high performance required \\
\multicolumn{4}{l}{\textbf{Layered}} \\
Virtual machines & Virtual machine offers services to the layers above it &
...many applications can be based upon a single, common layer of services

...interface service specification resilient when implementation of a layer
must change & ...many levels are required (causes inefficiency)

... data structures must be accessed from multiple layers \\
Client-server & Clients request service from a server &  ...centralised
computation and data at a single location (the server) promotes manageability
and scalability

...end-user processing limited to data entry and presentation &
...centrality presents a single point of failure risk

...network bandwidth limited

...client machine capabilities rival or exceed the server's \\
\multicolumn{4}{l}{\textbf{Dataflow styles}} \\
Batch-sequential & Separate programs executed sequentially with batched input &
...problem easily formulated as a set of sequential, severable steps &
...interactivity or concurrency between components necessary or desirable

...random access to data required \\
Pipe and filter & Separate programs (filters) executed, potentially concurrently.
Pipes route data streams between filters &  ...filters are useful in more than
one application

...data structures easily serialisable & ...interaction between components required \\
\end{tabular}
\end{table}

\begin{table}[H]
\begin{tabular}{p{0.2\textwidth} p{0.25\textwidth} p{0.25\textwidth}
p{0.25\textwidth}}
\textit{Style name} & \textit{Summary} & \textit{Use it when...} & \textit{Avoid it when...} \\ \hline
\multicolumn{4}{l}{\textbf{Shared memory}} \\
Blackboard & Independent programs access and communicate exclusively through
a global repository (the blackboard) & ...all calculations center on a common,
changing data structure

...order of processing dynamically determined and data-driven & 
...programs deal with independent parts of the common data 

...interface to common data susceptible to change

... interactions between the independent programs require complex regulation \\
Rule-based & Use rules in a knowledge base to resolve queries &
...problem data and queries expressible as simple rules over which inference may
be performed. & ...number of rules is large

...interaction between rules is present

...high-performance required \\
\multicolumn{4}{l}{\textbf{Implicit Invocation}} \\
Publish-subscribe & Publishers broadcast messages to subscribers & 
...components are very loosely coupled

...subscription data is small and efficiently transported &
...middleware to support high-volume data is unavailable \\
Event-based & Independent components asynchronously emit and receive
events communicated over event buses & ...components are concurrent and
independent

...components are heterogeneous and network-distributed & ...guarantees on
real-time processing of events is required
\end{tabular}
\end{table}

The selection of which style/pattern do use depends on a number of factors:
\begin{itemize}
\item Types of components/connects used in the pattern
\item Mechanisms by which control is shared, allocated and transferred among
the components
\item Ways of communicating data
\item Interaction of data and control
\item Invariants defined
\end{itemize}

\section{Lifecycle and Process}
\subsection{Architecture business cycle}
The architecture business cycle (ABC) is a cycle of influences between the
architecture of a system, and the technical, business and social aspects
of the system. This is typically captured in the functional and
non-functional requirements, but is also captured in its development
and target operating environments.

There are a number of business and social influences on a system outside
of the technical aspects:
\begin{itemize}
\item Stakeholders: - customers, users, developers, managers, ...
\item Developing organisation - Concerns for short term and long term
  business goals and organisational structure
\item Architects - The skill and experience of the architect will
  have a large influence on the system's architecture
\item Technical environment - The standard practices or techniques in
  place at a business can affect the technology or process choices
  when designing and implementing a software architecture
\end{itemize}

\subsection{Software development process}
The typical software development process involves
\begin{itemize}
\item Create business case
\item Understand requirements
\item Create or select architecture
\item Document and communicate architecture
\item Analyse or evaluate architecture
\item Implement system according to architecture
\item Check system conforms to the architecture
\item Maintain the system according to architectural constraints
\end{itemize}

To derive the architecture there are a few simple methods:
Reuse an existing architecture, develop following a method
or develop using intuition.

\textbf{Architecture reuse} \\
Take an architecture specification from an existing system
or literature. In theory, this is cheaper and quicker, but may
run into issues with portability and the taken architecture
not fitting the requirements or development environment.

\textbf{Develop by method} \\
A standard, systematic method could be used to develop a new
architecture. Methods typically involve deriving the architecture
from requirements through a series of transformations.
This is likely to be better documented, in terms
of both the architecture and the method. The choice of method
can be critical, as the wrong method can lead to a bad start
or incorrect priorities.

\textbf{Develop by intuition} \\
This is an interesting case where an experienced architect may
simply develop by their intuition without any methods or reuse.
With could lead to some innovative architectures, but comes at
an increased risk. It can also be potentially more optimal at a
higher cost. Evaluation on architectures developed by intuition
is critical to ensure the meet the requirements fully.

\subsection{Issues in architecting}
There are a number of issues that arise when developing a software
architecture. These can be split into the following categories
\begin{itemize}
\item \textbf{Decomposition} \\
  How are the sets of components/connectors and configurations
  derived and what constraints are imposed on these architecture
  elements? Decomposition goes down to the level of interactions
  and the degree of concurrency, but one must be mindful of what
  is a suitable level of granularity for decomposition, especially
  when only considering the architecture and not the implementation
\item \textbf{Composition} \\
  How do we determine what elements are needed and where do we find the
  required elements? Composition deals with what elements are needed,
  for example how to heterogeneity (different elements) and how do we
  know if the elements present are enough to achieve the target system.
\end{itemize}

There are also two approaches to tackle issues in development:
top-down or bottom up. Realistically a combination of both
approaches is used. An example of an architecture design method
is \textbf{attribute-driven design}. In this approach, the development
of the architecture is driven by quality attributes (performance,
reliability, safety, etc.). This is a recursive decomposition
method targeted at initial design. The idea is to identify tactics to
achieve the quality requirements.
\imagefig{0.6\textwidth}{imgs/attribute-driven.png}{Workflow for an
  attribute driven design of an architecture.}

Another example is with architecture definition activities
\begin{enumerate}
\item Consolidate inputs
\item Identify scenarios
\item Identify relevant architectural styles
\item Produce candidate architecture
\item Explore architecture options
\item Evaluation architecture with stakeholders
\item If not acceptable
  \begin{itemize}
  \item Rework architecture / revisit requirements
  \item Return to step 5
  \end{itemize}
\end{enumerate}

\section{Architecture views}
It is important that software architecture is documented in order
for it to be shared and understood for all its benefits to be realised.
Different stakeholders may be interested in different aspects of the
architecture, and a complicated architecture would require documentation
in different aspects for these stakeholders.

The main issue with software architecture documentation is the lack
of a universally accepted standard. Due to the complexity of architectures,
multiple views and levels of detail are often required. The different views
must remain consistency through the evolution of the architecture and remain
consistent among the different views.

\subsection{Concepts}
Even though documentation can exist, the question then becomes what should
be documented. Documentation that is too detailed will incur a high cost
for both the documentation process and the process for updating
it. Factors for choosing what to document include
\begin{itemize}
\item Level of detail
\item Longevity of the system
\item Needs of the stakeholders
\end{itemize}

The IEEE 1471 lists concepts which are important for documentation:
\begin{itemize}
\item Stakeholder - The person or organisation with an interest
  in the system
\item Concern - The interest of the stakeholder
\item View - Part of an architecture description
\item Viewpoint - Definition of a view (contents, models to use)
\item Model - Piece of structured information in a particular
  representation/language
\item Rationale - Explanation/motivation for the architecture description
\item Library viewpoint - Viewpoint that is to be used in
  different architecture descriptions
\end{itemize}

\begin{definition}
  A view is a representation of a coherent set of architecture
  elements written by and for stakeholders of the system
\end{definition}
A view is typically encapsulated by a set of design decisions
related by a common concern (or set of concerns). The idea of views
is to support different goals and uses. For example a view may be
captured in one or more models, which may be used in multiple views.

A viewpoint defines the perspective from which a view is taken.
In other words, a view is \textit{an instance of a viewpoint}
for a specific system. The important aspect of viewpoints is
that they taken into account which stakeholder's perspective when
looking at a view.
\subsubsection{Defining a viewpoint}
According to IEEE 1471, a viewpoint can be defined by the following
elements
\begin{itemize}
\item Name
\item Stakeholders
\item Concerns
\item Language, modelling technique, analytical methods
\item Source (reference)
\item Consistency/completeness checks
\item Evaluation/analysis techniques
\item Heuristics, tips for creating associated views
\end{itemize}

Example of viewpoints include logical, physical, deployment (
mapping of logical entities to physical entities), concurrency
and behaviour.

\subsection{4+1 viewpoints}
One example of multiple viewpoints is the 4+1 viewpoints by Kruchten.
Here the viewpoints are split into four different viewpoints:
\begin{itemize}
\item \textbf{Logical view} - The conceptual view which includes the
  structurally significant elements of the architecture such as the
  components and their interactions. Also known as a component and
  connector view
\item \textbf{Physical view} - The view that shows how logical
  structure is mapping onto application hardware
\item \textbf{Development view} - Shows the organisation of
  software in the development environment
\item \textbf{Process view} - Describes the concurrency and coordination
  elements of an architecture
\end{itemize}

\imagefig{0.6\textwidth}{imgs/4+1.png}{Diagram of the 4+1 viewpoints.}

The use cases (the +1) is used to relate all the views. It captures the
requirements for the architecture. It can be related to one or more
views and can be used to test and validate the architecture.

\subsection{C4 model}


\subsection{Other viewpoints}
There are many other viewpoints, for example
\begin{itemize}
\item Module view - A system structure as a set of implementation units
  with code modules, classes, packages and subsystems in the design
\item Component and connector view - A system structure as a set of
  units of run-time behaviour and interaction
\item Allocation view - A system structure in relation to development
  and execution environment
\end{itemize}

\section{Notations}
Notation is important to have well-defined and understood semantics.
There are different kinds of notations in which software
architecture may be expressed. 
\subsection{General purpose}
General purpose notations can be used in most circumstances, but do not
provide details in their notation. Example of general purpose notations
include UML and SysML.

\subsection{Architectural description language}
An architectural description language is one that can be used to describe
the architecture of a software system in terms of its elements and the
relationships and constraints among them. They can either be in graphical
or textual form. There are a great variety of existing ADLs that
support many different cases, such as domain-specific, static, dynamic
and interaction paradigms.

Examples of ADLs include
\begin{itemize}
\item Acme
\item Aesop
\item ArchWare
\item CommUnity
\item Darwin
\item Koala
\item Rapide
\item ...
\end{itemize}

\textbf{Darwin}
Darwin is a general purpose ADL for structures of distributed systems.


\section{Software architecture evaluation}
It is important to evaluate architectures. Architectural decisions are
important for multiple reasons:
\begin{itemize}
\item Blueprint for development and maintenance
\item Clarification and prioritisation of requirements
\item Development team structure and management decisions
\item Understanding, learning and sharing
\end{itemize}
An evaluation of the architecture helps deal with these decisions.
Further, it is cheaper to evaluate architectures if it is done
earlier in the lifecycle, as it minimises the cost of correcting
errors and unnecessary effort later. It is also easier to evaluate
the architecture rather than the implementation, as it is a higher
level model with only the features of interest. As such, architecture
evaluation provides benefits of increased quality, controlled cost
and reduction of risks.

\subsection{Ws}

\subsubsection{When?}
Often, it is best to evaluate an architecture as early as possible
and to incorporate the evaluation in the overall plan for the system.
It is also a good idea to evaluate regularly, just like unit testing
software. There is no bad time to evaluate an architecture, as even
an evaluation too late is useful to identify and resolve problems
that have been arising. For example, legacy systems require evaluation
to incorporate into new systems or to deal with maintenance issues.

\subsubsection{Who?}
Typically the development team itself will carry out the evaluation, but
it can also be a different team within the same organisation or an
independent/external reviewer.

The main issues that can arise from who is involved with the evaluation
is bias, either from the same development team or a rival
development team. Moreover, it is important for reviews to have a
good amount of domain knowledge and experience.

\subsubsection{What?}
\begin{itemize}
\item Architecture process
\item Architecture description
\item Overall issues
\item Specific concerns
\item Future changes and impact
\end{itemize}

\subsubsection{How?}
\begin{itemize}
\item Questioning techniques - Generate a discussion based on
  qualitative questions, questionnaires, checklists and scenarios
\item Measuring techniques - Provide quantitative answers to
  specific questions, metrics, simulation, prototypes and formal
  analysis
\end{itemize}


\subsection{Dimensions of evaluation}
\begin{itemize}
\item Goals - Completeness, consistency, compatibility,
  correctness
\item Scope
  \begin{itemize}
  \item Component and connector
  \item Subsystem and system level
  \item Data exchange
  \item Comparison of architectures
  \end{itemize}
\item Concerns - Structure, behaviour, interaction,
  non-functional
\item Models - Informal, semi-formal, formal
\item Type - static, dynamic and scenario-based
\item Automation level - manual, partially automated and
  fully automated
\item Stakeholders
\end{itemize}

\subsubsection{Characteristics of successful evaluation}
\begin{itemize}
\item Clear set of goals and requirements
\item Well-defined scope (small number of goals)
\item Cost-effectiveness
\item Availability of required personnel and architectural models
\item Competent evaluation team
\item Well-defined and realistic expectations
\end{itemize}

\subsection{Architecture Tradeoff Analysis Method (ATAM)}
ATAM is a scenario-based evaluation method that focuses on tradeoffs
between quality goals. It requires the participation of three
different groups:
\begin{enumerate}
\item The evaluation team (external to the project)
\item Project decision makers (including the architect)
\item Architecture stakeholders (developers, testers, users)
\end{enumerate}

The ATAM process involves 2 phases with a few steps:
\begin{itemize}
\item Phase 1 (groups 1 and 2)
  \begin{itemize}
  \item Present the ATAM
  \item Present business drivers
  \item Present architecture
  \item Identify architectural approaches and patterns used
  \item Generate quality attribute utility tree
  \item Analyse architectural approaches
  \end{itemize}
\item Phase 2 (groups 1, 2 and 3)
  \begin{itemize}
  \item Brainstorm and prioritise scenarios
  \item Analyse architectural approaches
  \item Present results
  \end{itemize}
\end{itemize}

%\imagefig{}{}{} ATAM utility tree

ATAM also has a set of deliverables at the end of the process.
\begin{itemize}
\item Concise presentation of architecture
\item Business goals
\item Quality requirements in terms of scenarios
\item Mapping of architectural decisions to quality requirements
\item A set of identified sensitivity and tradeoff points
\item A set of identified risks and overarching risk themes
\end{itemize}

\subsection{Metrics}
\begin{itemize}
\item Size of the architecture
\item Connectivity and complexity of architecture
\item Coupling and cohesion among architectural elements
\item Changes made to architecture
\end{itemize}



%\printbibliography

\end{document}



