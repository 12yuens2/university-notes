\documentclass[Report.tex]{subfiles}
\begin{document}
\section{Morphology}
Morphology is the study of the ways in which words are built up of
smaller \textbf{morphemes}. There are two general units of morphemes:
\begin{itemize}
\item \textbf{stems} - the core meaning-bearing units
\item \textbf{affixes} - bits and pieces that adhere to stems to change
their meanings and grammatical functions
\end{itemize}
For example, the word ``unbelievably'' can be broken down into the
stem `believe' and affixes un-, -able and -ly.

Affixes come in many forms:
\begin{itemize}
\item \textbf{prefix} - precedes the stem (un-)
\item \textbf{suffix} - follow after the stem (-able)
\item \textbf{infix} - inserted into the stem
\item \textbf{interfix} - linking two words together (speed -o- meter)
\item \textbf{circumfix} - one part before and one part after
\end{itemize}
English has no real infixes or circumfixes, and there are no real infixes
in other European languages either.


\subsection{Combining morphemes}
There are a few ways of combining morphemes together
\begin{itemize}
\item \textbf{inflection} - combining the stem with a grammatical
  morpheme, resulting in a word of the same class.
  This may sometimes involve agreement. (e.g. I walk, he walks)
\item \textbf{derivation} - creating a word in another class
  (e.g. computer $\rightarrow$ computerise $\rightarrow$ computerisation)
\item \textbf{compounding} - combining stems (e.g. dog + house $\rightarrow$ doghouse)
\item \textbf{clitisation} - attaching a clitic to the stem. A clitic is
  a morpheme that acts as a word, but is typically reduces in form
  and is attached to another word (e.g. we've)
\end{itemize}

\subsubsection{Inflection}
Inflection is mostly productive, i.e. applies to (almost) all words in a class.
Examples of this include verbs in -s form, or -ing participle, plurals
or nouns etc. However, there are some irregular forms such as
catch $\rightarrow$ caught. It is important to remember inflection does
not change the class of a word, only the grammar or agreement.

\subsubsection{Derivation}
In derivation, the class of the words are changed.
\begin{itemize}
\item verb into noun (kill $\rightarrow$ killer, -er)
\item noun into verb (digit $\rightarrow$ digitise, -ise)
\item noun into adjective (fur $\rightarrow$ furry, -r)
\item adjective into noun (fuzzy $\rightarrow$ fuzziness, -iness)
\item adjective into adverb (quick $\rightarrow$ quickly, -ly)
\end{itemize}
Note most of these derivations are suffix morphemes. Further,
derivations are often not productive, results in meaning that
is not always predictable.

\subsubsection{Compounding}
Where possible morphology is seen as independent from conventions
of spelling. For example, ``sunglasses'', ``life-threatening''
and ``football stadium'' are all examples of compounds, even though
the connection or interfix does not always exist.

In practical systems, the distinctions between morphology and
orthography is often blurred, for obvious reasons.

\subsubsection{Clitisation}
There are two types of clitics:
\begin{itemize}
\item \textbf{proclitic} - before the main word
\item \textbf{enclitic} - after main word
\end{itemize}

In English, the `an' in `an apple' is sometimes considered to
be a proclitic.

\subsubsection{Non-concatenative morphology}
Most of the above examples are \textbf{concatenative morphology},
where words are built up of morphemes one after the other.
In some other languages (e.g. semitic languages) there is a
\textbf{template-morphology} instead using a root-and-pattern
morphology.

For example, in Hebrew the root \textbf{l-m-d} (three consonants
$C_1-C_2-C_3$ means study. The meaning is then changed by
inserting affixes between the root.
\begin{table}[H]
\begin{tabular}{l l l}
  \textbf{pattern} & \textbf{form} & \textbf{meaning} \\
  $C_1$a$C_2$a$C_3$ & lamad & he studied \\
  $C_1$i$C_2$e$C_3$ & limed & he taught \\
  $C_1$u$C_2$a$C_3$ & lumad & he was taught \\
  $C_1$a$C_2$an$C_3$ & lamdan & scholar 
\end{tabular}
\end{table}

\subsection{Agreement}
In English, there is \textbf{agreement} in \textbf{number}
(singular/plural) and \textbf{person} (1st, 2nd, 3rd) between
the subject and verb. For example
\begin{itemize}
\item I \underline{am} at the fair (1st person singular)
\item She \underline{is} at the fair (2nd person singular)
\item She and her friend are at the fair (2nd person plural)
\end{itemize}
Other languages also have agreement in \textbf{case}
(nominative, genitive) and/or \textbf{gender}
(masculine/feminine in French and Italian). Gender classes of nouns
may be detached from meanings.

\subsection{Finite-state lexica}
A simple approach to constructing a FSA accepting words in a
language is to use morphemes.
\begin{enumerate}
\item Model the morphotactics, i.e. which classes of morphemes
  can follow morphemes of which other classes
\item Compile lists of classes of those morphemes
\item Combine the two, substituting the morphemes for classes
  in the FSAs
\item Take union of all such FSAs (and determinise for efficiency)
\end{enumerate}
A FSA can be constructed to distinguish between nouns having regular
plurals (reg-noun) and irregular ones, for which we separately
list singulars and plurals.

\imagefig{0.6\textwidth}{imgs/fsa-lexicon.png}{Morphotactics as FSA}

Similarly, morphotactics can be used for verb forms and other inflectional
morphology. Derivational morphology is more complex. The FSAs can also be
generalised to FSTs to output information about which word was accepted
with which features (e.g. number, gender, verb tense, etc.)

\subsection{Morphological parsing}
Morphologies can be parsed by dividing a word into its morphemes, and
obtaining grammatical properties of those morphemes. For example,
cats $\rightarrow$ cat + N + PL.

Sometimes, the mapping directly can be difficult to model,
so multiple tapes need to be used:
\begin{itemize}
\item \textbf{Lexical form} - fox + N + PL
\item \textbf{Intermediate form} - fox\^{}s\#
\item \textbf{Surface form} - foxes
\end{itemize}
Here the carot \^{} stands for a morpheme boundary and \# stands
for a word boundary.

The mapping form lexical form to intermediate form is relatively
straightforward using a FST. The mapping from intermediate form
to surface form can be handled by rewriting some rules, such as
\textit{insert an `e' between \{x, s, z\}\^ and s\#}. This rule
can then be compiled to FST. For several orthographic rules,
we get several FSTs, which can be combined into a single FST.
Further composition with FST for the lexical/intermediate mapping.
The result is typical cascade architecture.
\end{document}
