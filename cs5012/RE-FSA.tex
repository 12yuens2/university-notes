\documentclass[Report.tex]{subfiles}
\begin{document}

\section{Regular expressions and automata}
Regular expressions are a powerful way to search texts in strings. Most
programming languages support regular expressions natively in some way.
They work by searching for patterns and replacing the pattern with another.

For example, the following searches can be performed on the word `woodchuck`:
\begin{itemize}
\item \texttt{/d/} matches the \texttt{d} in woodchuck
\item \texttt{/s/} does not matching anything in woodchuck
\item \texttt{/ck/} matches the last two letters in woodchuck
\end{itemize}

\subsection{Basic regular expression patterns}
Regular expressions are case sensitive, so a lowercase \texttt{/s/} will not
match words with an upper case S. Square brackets can be used to specify
a \textbf{disjunction} of characters to match. For example, the \texttt{/[wW]/}
pattern will match either an upper or lower case w. A dash (\texttt{-}) can
also be used to specify any one character in a range. For example, 
\texttt{/[2-5]/} specifies any one of the characters 2, 3, 4 or 5.
\begin{table}[H]
\centering
\begin{tabular}{l l l}
\textbf{RE} & \textbf{Match} & \textbf{Example pattern} \\
\texttt{/[wW]oodchuck/} & Woodchuck or woodchuck & ``\underline{Woodchuck}" \\
\texttt{/[abc]/} & `a', `b' or `c' & ``In uomini, in sold\underline{a}ti" \\
\texttt{/[1234567890]/} & a digit & ``plenty of \underline{7} to 5" \\
\texttt{/[0-9]/} & a digit & ``plenty of \underline{7} to 5" \\
\texttt{/[A-Z]/} & an uppercase letter & ``we should call it \underline{Drenched}"
\end{tabular}
\caption{Example regular expressions using square brackets \texttt{[ ]}.}
\end{table}
Here are a few basic rules and expressions:
\begin{itemize}
\item \texttt{/[\^{}abc]/} matches any character other than a, b or c using the caret `\texttt{\^{}}'\\
\item \texttt{/./} is a \textbf{wildcard} that matches any character
\item \texttt{/[abc]+/} matches one or more of a, b or c
\item \texttt{/[0-9]*/} matches zero or more digits
\end{itemize}
There are also a number of aliases for sets of characters that can be used
as a shortcut.
\begin{itemize}
\item \texttt{\textbackslash d} for any digit (same as \texttt{[0-9]})
\item \texttt{\textbackslash D} for any character other than a digit 
(same as \texttt{\^{}[0-9]})
\item \texttt{\textbackslash w} for any alphanumeric or underscore
\item \texttt{\textbackslash W} converse of the above
\item \texttt{\textbackslash s} for any whitespace character (spaces and tabs)
\item \texttt{\textbackslash S} converse of the above
\end{itemize}
Repetition can be matched using curly brackets
\begin{itemize}
\item \texttt{/\textbackslash d\{2,5\}/} matches between 2 and 5 digits
\item \texttt{/\textbackslash s\{3,\}/} matches 3 or more whitespaces
\item \texttt{/\textbackslash d\{4\}/} matches exactly 4 digits
\end{itemize}
\textbf{Anchors} can be used to match places in a string rather than characters.
For example, \texttt{\textbackslash b} can match word boundaries so that
an expression such as \texttt{/\textbackslash bthe\textbackslash b/} does
not match the three letters \texttt{the} inside \texttt{other}.
\begin{itemize}
\item \texttt{/\^{}/} matches beginning of input string or beginning of line
\item \texttt{/\$/} matches end of input string or end of line
\item \texttt{/\textbackslash b/} matches a word boundary
\end{itemize}
The power of regular expressions can be captured by a small number of operators
for concatenation, union and Kleene closure. This defines the class of 
\textbf{regular languages}.

\subsection{Regular languages}
The class of regular languages is formally defined as:
\begin{enumerate}
\item $\emptyset$ and \{$\epsilon$\} are regular languages ($\epsilon$ stands for
the \textbf{empty string})
\item \{$a$\} is a regular language for any symbol $a$ in the `alphabet'
\item If $L_1$ and $L_2$ are regular languages, then so are:
	\begin{enumerate}
	\item $L_1$ \cdot $L_2$ = \{$xy | x \in L_1, y \in L_2$\} (concatenation)
	\item $L_1$ \cup\ $L_2$ (union/disjunction)
	\item $L_{1}*$ (Kleene closure)
	\end{enumerate}
\end{enumerate}
Nothing else is a regular language unless it can be obtained through the above rules.
Regular languages are the languages characterisable by regular expressions,
so all regular expression operators can be implemented by the three operations
which define regular languages.

Regular languages are closed under the following operations
\begin{itemize}
\item \textbf{intersection} - if $L_1$ and $L_2$ are regular languages, then so is
$L_1$ \cap\ $L_2$, the language consisting of the set of strings that are in
\textit{both} $L_1$ and $L_2$.
\item \textbf{difference} - if $L_1$ and $L_2$ are regular languages, then so is
$L_1 - L_2$, the language consisting of the set of strings that are in $L_1$
but not in $L_2$.
\item \textbf{complementation} - if $L_1$ is a regular language, then so is
$\sum *$ where $\sum$ is the infinite set of all possible
strings formed by the alphabet $\sum$ and $\sum *$ is the set of all
possible strings that aren't in $L_1$.
\item \textbf{reversal} - if $L_1$ is a regular language, then so is $L_1^R$,
the language consisting of the set of reversals of all the strings in $L_1$
\end{itemize}
The proof that regular expressions are equivalent to finite-state automata 
has two parts: showing than an automaton can be built for any regular expression
and showing that a regular language can be built for any automaton.

\subsection{Finite-state automata}
A finite automata (FA) consists of:
\begin{itemize}
\item a finite set $Q$ of \textbf{states}
\item a finite set $\sum$ of input symbols (\textbf{alphabet)}
\item an \textbf{initial state} $q_0 \in\ Q$
\item a set $F \subseteq Q$ of \textbf{final states}
\item a set $\Delta$ of \textbf{transitions}
\end{itemize}
\imagefig{0.75\textwidth}{imgs/fsa.png}{Example of a finite-state automaton with
an epsilon transition.}
A transition has the form $(q, a, q')$ which says, go to state $q'$ from $q$
by reading the next symbol $a$. An epsilon transition $(q, \epsilon, q')$ 
allows a transition between the states without reading an input symbol.

A string of input symbols is \textbf{accepted} by an automaton if it can
go from an initial state to a final state by reading all the symbols
left to right and following a valid sequence of transitions. Therefore,
the language accepted by a finite-state automaton is the set of strings
that is accepted by it.

\subsubsection{Deterministic vs. non-deterministic}
The difference between deterministic and non-deterministic finite automata
is the existence of epsilon transitions and valid transitions - that is
to say, for each $q$ and $a$ there is at most one $q'$ such that
$(q, a, q') \in \Delta$.
\imagefig{0.75\textwidth}{imgs/nfsa.png}{Example of a non-deterministic
finite-state automaton. In $q2$ when the symbol $a$ is read, it can
either move to $q2$ or $q3$.}

It is not the case that non-deterministic finite-state automata are more
powerful than deterministic automata. For any NFSA, there is exactly one
equivalent DFSA, although the number of states in the equivalent DFSA
may be much larger. This is done through the construction of a 
powerset, where each state of $M'$ represents a set of states of $M$.

\subsubsection{Recognition}
Assuming an input of length $n$. An input position is a number from 0 to $n$.
\begin{itemize}
\item A \textbf{configuration} is defined to be a pair consisting
of a state and an input position
\item The \textbf{initial configuration} is $(q_0, 0)$
\item A configuration $(q, i)$ is \textbf{accepting} if $q \in F$ and $i = n$
\item The \textbf{agenda} is a set of configurations, which is initially empty.
\end{itemize}
Recognition can be run by adding and remove configurations from the agenda on
every iteration, then if the agenda ever becomes empty, we have run out
of valid configurations, so the recognition can stop and return a failure.
If the agenda ever contains an accepting configuration, then we can stop
and return a success.

\subsubsection{From NFSA to regular expression}
For each regular expression, we can construct an equivalent finite automata.

\textbf{Proof} by induction:
\begin{itemize}
\item Show that the FSAs for $\emptyset$, \{$\epsilon$\} and \{$a$\} for 
every symbol $a$ can be constructed.
\item Show that if we have FSAs accepting $L_1$ and $L_2$, then we can
construct three FSAs accepting $L_1 \cdot L_2$, $L_1 \cup\ L_2$ and $L_1 *$
respectively.
\end{itemize}

Another important point for closure is that FSA languages are
closed under concatenation, union and Kleene star.

Similarly, for each FSA, we can construct a regular expression describing
the same language with a divide-and-conquer approach. This is more
tricky than converting from a regular expression to NFSA.

TODO NFSA to RE example

Furthermore, the class of regular languages is closed under:
\begin{itemize}
\item intersection ($L_1 \cap\ L_2$)
\item complementation ($\sum *  - L_1$)
\item set difference ($L_1 - L_2$)
\item reversal (replace every string by its mirror image)
\end{itemize}

\subsection{Finite-state transducers}
Finite-state transducers are very similar to FSAs, but instead of using a single
symbol for transitions, a pair of strings ($v$, $v'$) is used, in the input
and output alphabets respectively.

This means each transition now takes the form $(q, v, v', q')$ where
\begin{itemize}
\item $v$ is a string over the input alphabet
\item $v'$ is a string over the output alphabet
\end{itemize}
The transducer goes from state $q$ to $q'$ by reading $v$ from the input
and writing $v'$ to the output.
\imagefig{0.4\textwidth}{imgs/fst.png}{Example finite-state transducer.}

FSTs can be seen as reading a pair of strings simultaneously, or as 
generated two strings simultaneously, or translating from input into output. 

Importantly, the class of rational transductions is \textbf{not} closed under
intersection, but they are closed under \textbf{inversion} (swap input
and output parts of all transitions).

\subsubsection{Sequential transducer}
A sequential transducer is a special case FST such that:
\begin{itemize}
\item every transition has the form $(q,a,v,q')$ for a symbol $a$
\item given $q$ and $a$, there is at most one pair $(v, q')$ such that
$(q, a, v, q')$ is a transition
\end{itemize}
In other words, there is only one or less transitions for each $(q, a)$ pair.
It behaves like a DFA on the input side. If the output was ignored,
it would be exactly a DFA.

\subsection{Minimum edit distance}
The minimum edit distance describes the minimum number of insertions,
deletions and substitutions needed to turn one string into another.
This has useful applications in error correction.

For example, the minimum edit distance between the words \textsc{INTENTION}
and \textsc{EXECUTION} is 5:
\begin{table}[H]
\centering
\begin{tabular}{c c c c c c c c c c}
I & N & T & E & * & N & T & I & O & N \\
\_ & \_ & \_ & \_ & \_ & \_ & \_ & \_ & \_ & \_ \\
* & E & X & E & C & U & T & I & O & N \\
d & s & s &   & i & s &   &   &   &   \\
\end{tabular}
\end{table}

To compute this an be sure there is no better solution:
\begin{itemize}
\item let strings $v = a_1 ... a_m$ and $w = b_1 ... b_n$
\item let $0 \leq i \leq m$ and $0 \leq j \leq n$
\item let $dist(i, j)$ be the minimum edit distance of $a_1 ... a_i$ 
and $b_1 ... b_j$
\end{itemize}
So the minimum edit distance of $v$ and $w$ is $dist(n, m)$ and we can
clearly see that $dist(0, 0) = 0$.

Now for $(i,j) \neq (0,0), dist(i, j)$ is the minimum of the following:
\begin{itemize}
\item $dist(i - 1, j - 1) + s(a_i, b_j)$ only if $i \geq 1$ and $j \geq 1$
\item $dist(i, j - 1) + i(b_j)$ only if $j \geq 1$
\item $dist(i - 1, j), + d(a_i)$ only if $i \geq 1$
\end{itemize}
We can assume the cost of insertion and delete is 1 for all symbols
and the cost of substitution is 0 if $a = b$ and 1 otherwise.


\subsection{Tokenisation}
Tokenisation is the idea of dividing text into words, punctuation, numbers, etc.
It is often the first phase in analysis of written text. There are many
difficulties in doing this, such as a period `.' as it can have multiple
meanings (end of sentence, abbreviation, decimal point, etc.). We can
see finite-state technology to solve most problems of tokenisation.

This is more complex in languages that do not separate words by whitespace,
such as Chinese and Japanese. A typical approach to tokenisation for
these languages is to use an `eager' dictionary method by matching the
longest sequences of characters left-to-right that can be found in the
dictionary. When a match is found, remove the characters from the text and
continue match. This is called \textbf{MaxMatch} (maximum match). The main
weakness of this approach is if there are unknown words (not in the dictionary)
in the text.


\end{document}