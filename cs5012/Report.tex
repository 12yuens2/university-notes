\documentclass{sty/SizheArticle}

\title{Lecture Notes}
\author{Sizhe Yuen}
\addbibresource{references.bib}
\usepackage{sty/sizhetitle}
\usepackage{amsmath, amsfonts}
\usepackage{dsfont}
\usepackage{subfiles}
\begin{document}


\maketitlepage{CS5012 Language and Computation}{Mark-Jan Nederhof}
\tableofcontents
\clearpage

\section{Introduction}
Natural language processing is the idea of getting computers to perform interesting
tasks involving human languages. The goal is to provide insights in
how languages are processed.

\subsection{Motivation}
There is a wealth of written knowledge, especially on the Web in a machine readable
form. To be able to process and understand this knowledge would be greatly
beneficial. By improving machine processing of language, both human-machine
communication and human-human (translation) communication can be improved.

\subsubsection{Example applications}
Small scale:
\begin{itemize}
\item Spelling correction
\item Hyphenation
\end{itemize}

Medium scale:
\begin{itemize}
\item Web analytics - data mining web pages
\end{itemize}

Large scale:
\begin{itemize}
\item Machine translation
\item Text summarisation
\item Question/answer conversation
\end{itemize}

\subsection{Topics}
\begin{itemize}
\item Phonology - how sounds make speech
\item Morphology - components of words
\item Syntax - how to put words together
\item Semantics - meaning of words and sentences (lexical and compositional 
semantics)
\item Pragmatics - goals and intentions of the speaker
\item Discourse - how to put utterances together to give meaningful communication
\end{itemize}

\subsection{Ambiguity}
A major problem that all language processing tasks must try to solve is
ambiguity at some level. An input is \textbf{ambiguous} if \textit{there are
multiple alternative linguistic structures that can be built for it}.
Humans generally resolve ambiguity naturally and are unaware of the ambiguity
through the use of context, world knowledge, common sense, etc.

For example, the sentence ``\textit{I made her duck}" can have multiple meanings:
\begin{itemize}
\item I cooked waterfowl for her (to eat)
\item I cooker waterfowl belonging to her
\item I created the (plaster?) duck she owns
\item I caused her to quickly lower her head or body
\item I waved my magic wand and turned her into undifferentiated waterfowl
\end{itemize}
This happens as each word in the sentence can have multiple meanings.
\begin{itemize}
\item \textbf{duck} - can be either a noun or a verb
\item \textbf{her} - can be possessive or dative pronoun
\item \textbf{make} - can mean `create', `cook' or `force'
\item \textbf{make} - can be used as a transitive or ditransitive verb
\end{itemize}
Models and algorithms used in language processing often try to resolve or
disambiguate these cases. For example, deciding whether `duck' is a noun or a
verb can be solved by \textbf{part-of-speech tagging}. This is a kind of
\textbf{lexical disambiguation}. In contrast, deciding whether `her' and `duck'
are part of the same entity is an example of \textbf{syntactic disambiguation},
which can be addressed by probabilistic parsing.


\subfile{RE-FSA}
\subfile{morphology}

\end{document}



