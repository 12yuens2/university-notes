\documentclass[CS4204-Notes.tex]{subfiles}
\begin{document}

\section{Evaluation strategies}
Evaluation strategies use lazy higher-order functions to separate the two concerns of specifying the algorithm and specifying the program's dynamic behaviour. A function definition can then be split into two parts: the algorithm and the strategy.
\begin{lstlisting}[caption={The strategy type specifies the ``evaluation'' type.}]
type Strategy a = a -> Eval a
\end{lstlisting}
The separation of concerns makes both the algorithm and the dynamic behaviour easier to comprehend and modify. Changing the algorithm may entail specifying new dynamic behaviour and conversely, it is easy to modify the strategy without changing the algorithm. Strategies have several desirable properties:
\begin{itemize}
\item Strategies are powerful - simple strategies can be composed, or passed as arguments to form more elaborate strategies
\item Strategies can be generic over all types in a language
\item Strategies are extensible - allowing a user to define new application-specific strategies
\item Strategies are type safe - the normal type system applies to strategic code
\end{itemize}
It should be possible to understand the meaning of a function without considering its behaviour. Strategies also help fight the laziness which only evaluates when needed.
\n
The idea of strategies is to abstract the algorithm from the behaviour, for example different strategies can be applied to the same algorithm that results in different parallel behaviour.
\begin{lstlisting}[caption={Evaluate \texttt{qsort xs} completely}]
qsort xs `using` rdeepseq
\end{lstlisting}
\begin{lstlisting}[caption={Evaluate \texttt{qsort xs} as a data parallel computation}]
qsort xs `using` parList rdeepseq
\end{lstlisting}

\subsection{Basic evaluation strategies}
There are three most basic strategies which specify simple evaluation:
\begin{itemize}
\item \textbf{r0} - no evaluation
\item \textbf{rseq} - evaluate the expression
\item \textbf{rpar} - spark the expression
\end{itemize}
\begin{lstlisting}[caption={Definitions of basic strategies}]
  r0 :: Strategy a
  r0 x = return x

  rseq :: Strategy a
  rseq x = x `pseq` return x

  rpar :: Strategy a
  rpar x = x `par` return x
\end{lstlisting}
Strategies all do some kind of evaluation and return a result. The \texttt{r0} strategy does nothing as it is there for the sake of the type system to specify a strategy that does \textit{no} evaluation.

\subsection{Using strategies}
The \texttt{\textquoteback using\textquoteback} is a keyword used to apply a strategy, for example
\begin{lstlisting}
  using :: a -> Strategy a -> a

  f x `using` rpar --spark f x
  f x `using` rseq --evaluate f x
\end{lstlisting}
The line \texttt{e \textquoteback using\textquoteback strat} specifies that expression \texttt{e} uses strategy \texttt{strat}.
\n
The \texttt{withStrategy} function can also be used to apply strategies. It is defined as
\begin{lstlisting}
  withStrategy :: Strategy a -> a -> a

  --Example
  withStrategy rseq (qsort xs)
\end{lstlisting}
It evaluates \texttt{e} with strategy \texttt{strat} and is most useful for building new strategic functions and skeletons
\begin{lstlisting}
spark = withStrategy rpar
\end{lstlisting}

\subsection{Evaluation}
The \textbf{Eval} type wraps up values and is defined as a monad
\begin{lstlisting}
  data Eval a = Done a

  instance Monad Eval where
    return = Done
    m >>= k = case m of
      Done x -> k x

  -- Puts values into the monad
  return :: Eval a
  
  -- Sequences two Eval operations
  (>>=) :: Eval a -> (a -> Eval b) -> Eval b
  
  -- Extracts results from the monad
  runEval :: Eval a -> a
\end{lstlisting}
In Haskell, the evaluation is \textbf{lazy}, which means it only evaluates what is needed for a result. For data structures like lists, this means only the parts of the structure that are needed are created, which allows infinite structures to be created. By default, Haskell uses the \texttt{rseq} strategy for evaluation.

\subsubsection{Deep evaluation}
To force evaluation as part of a strategy, we need to evaluate the entire result, for example if we wanted the entire list result. To do this, the \texttt{rdeepseq} strategy is needed. \texttt{rdeepseq} forces the evaluation when needed, which fully evaluates the expression.
\begin{lstlisting}
  rdeepseq :: NFData a => Strategy a
  rdeepseq x = rnf x `pseq` return x
\end{lstlisting}
It is usually better than \texttt{rseq} but may evaluate twice. Further, it forces eager evaluation.

\subsubsection{NFData class}
The \texttt{NFData} class is a Haskell type class which incldues the \texttt{rnf} operations, whose purpose is to fully evaluate its argument. It is defined using \texttt{seq} but does not guarantee the evaluation happens in order.
\begin{lstlisting}
  class NFData a where
    rnf :: a -> ()
    rnf x = x `seq` ()
\end{lstlisting}

\subsubsection{Strategy composition}


\end{document}